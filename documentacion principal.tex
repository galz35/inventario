Eres Analista Funcional + Arquitecto de Software + DBA SQL Server + UX Lead + Líder Técnico.
Tu trabajo es generar un DOCUMENTO ÚNICO en español, extremadamente detallado, llamado:

“Documento de Implementación — Sistema de Inventario y Órdenes de Trabajo”

Reglas:
- Títulos simples (sin palabras decorativas).
- Lenguaje claro y directo.
- Diseño simple + eficiente + preciso + bonito (blanco/gris/negro + rojo + verde; NO usar azul).
- Todo movimiento debe ser trazable y auditable.
- Inventario exacto: prohibido editar stock directo.
- Debes entregar: especificación funcional + modelo de datos SQL Server + SP/consultas base + reportes + pantallas + checklist 0–100.
- Si hay algo incierto, ponlo en “Pendiente por confirmar” (no inventes).

Contexto del negocio:
- Empresa de servicios (telecom HFC) que atiende clientes por CASOS/ÓRDENES DE TRABAJO (OT) y también PROYECTOS.
- No vende productos al público. Los materiales se consumen en OT/proyectos.
- Maneja:
  1) Inventario consumible multi-almacén (con almacén padre y sub-almacenes).
  2) Inventario consignado de proveedores (propiedad proveedor).
  3) Transferencias entre almacenes (con confirmación).
  4) Stock en camioneta del técnico (almacén tipo TÉCNICO).
  5) Conteo físico vs sistema (con aprobación y aplicación).
  6) Ajustes (solo por conteo o motivo documentado).
  7) Merma/daño/pérdida.
  8) Activo fijo serializado: equipos y herramientas (asignación, devolución, reparación, baja).
  9) Equipos serializados instalados en cliente (activo instalado, retiro, reemplazo).
  10) Seguimiento de OT (estado, checklist, evidencia, firma, tiempos).
  11) Seguimiento de proyecto (estado, etapas/tareas, avance, consumo, costos).

Stack técnico:
- Backend: NestJS REST (o API equivalente)
- DB: SQL Server (principal)
- Frontend: React/Next responsive (móvil y web en un solo sistema)
- Auth: JWT + Refresh + RBAC (roles/permisos)

========================================================
SECCIÓN 0 — RESUMEN (1 página)
========================================================
Incluye:
- Objetivo del sistema
- Usuarios y roles
- Módulos
- Beneficios operativos
- Fuera de alcance (por defecto: facturación, caja, POS, compras completas)

========================================================
SECCIÓN 1 — GLOSARIO (OBLIGATORIO)
========================================================
Define términos con ejemplos:
- Almacén padre / sub-almacén
- Kardex
- Movimiento
- Consignación
- OT (Orden de trabajo) / Caso
- Proyecto
- Consumo
- Activo fijo (serializado)
- Equipo instalado (serializado)
- Conteo físico
- Ajuste / Merma / Pérdida
- Confirmación de transferencia

========================================================
SECCIÓN 2 — ROLES, PERMISOS Y LOGIN (OBLIGATORIO)
========================================================
2.1 Roles mínimos:
- Administrador
- Bodega
- Supervisor
- Técnico
- Gerencia
- (Opcional) Backoffice/Despacho (quien asigna OT)

2.2 Matriz permisos por módulo/acción:
Acciones: VER, CREAR, EDITAR, ELIMINAR, APROBAR, EXPORTAR
Módulos: Usuarios/Roles, Catálogos, Almacenes, Inventario, Kardex, Transferencias, Consignación, OT, Proyectos, Activos, Conteo, Reportes, Parámetros.

2.3 Reglas de acceso:
- Técnico solo ve: sus OT, su camioneta (almacén), activos asignados y consumo propio.
- Gerencia solo lectura reportes.
- Ajustes/aplicar conteo requiere APROBAR.
- Baja de activo requiere APROBAR.

2.4 Seguridad:
- bcrypt
- JWT (15-30 min)
- Refresh token (7-30 días) guardado hasheado
- logout revoca refresh
- usuario activo/inactivo
- auditoría (created_at/by updated_at/by)
- bitácora de acciones (tabla de auditoría de eventos)

Checklist + DoD.

========================================================
SECCIÓN 3 — MÓDULOS Y CÓMO FUNCIONA CADA UNO
========================================================

3.1 Catálogos
- Productos consumibles (por cantidad)
- Productos serializados (se manejan como activos/equipos; no por cantidad)
- Categorías
- Proveedores
- Motivos de movimiento
- Clientes (mínimo: nombre, teléfono, dirección)
- Tipos de OT (instalación, mantenimiento, reparación) con reglas de evidencia/firma

Reglas:
- Si producto es serializado => no aparece para “consumo por cantidad”, solo como activo/equipo.

3.2 Almacenes (multi-almacén con padre)
- Tabla almacenes con jerarquía: idPadre
Tipos:
- CENTRAL, REGIONAL, PROYECTO, TECNICO
Reglas:
- Un almacén TECNICO pertenece a un usuario técnico.
- Un almacén PROYECTO pertenece a un proyecto.

3.3 Inventario consumible (exacto)
Regla máxima:
- PROHIBIDO editar stock directo.
- Todo cambio es un movimiento kardex, con transacción.

Tipos de movimiento consumible:
- ENTRADA_COMPRA (solo registrar)
- ENTRADA_CONSIGNACION
- TRANSFERENCIA_SALIDA / TRANSFERENCIA_ENTRADA (generadas al confirmar)
- CONSUMO_OT
- DEVOLUCION_OT (de técnico a almacén)
- DEVOLUCION_PROVEEDOR (consignación)
- MERMA_DAÑO
- AJUSTE_CONTEO
- AJUSTE_MANUAL (solo Admin/Sup con motivo y aprobación)

3.4 Transferencias entre almacenes (con confirmación)
Flujo:
1) Crear transferencia (PENDIENTE) con detalle productos/cantidades/propietario (EMPRESA/PROVEEDOR)
2) En tránsito (opcional)
3) Confirmar recepción (CONFIRMADA) => genera 2 movimientos en kardex en una transacción:
   - salida origen
   - entrada destino
4) Cancelar (CANCELADA) sin afectar stock

Reglas:
- Si no se confirma, stock no cambia.
- Confirmación solo por rol autorizado.

3.5 Consignación (propiedad proveedor)
Regla:
- Stock consignado se separa por propietario=PROVEEDOR + proveedorId.
Flujos:
- Entrada consignación
- Consumo consignado en OT
- Devolución a proveedor
- Liquidación por período (consumo - devoluciones)

3.6 OT (Órdenes de trabajo / Casos)
Campos:
- cliente
- dirección
- tipo OT
- prioridad
- estado
- técnico asignado
- fechas (asignación, inicio, cierre)
- checklist (simple)
- evidencias (fotos)
- firma (opcional según tipo)
- consumos (materiales)
- activos instalados/retirados/reemplazados

Estados OT sugeridos:
- REGISTRADA
- ASIGNADA
- EN_PROCESO
- EN_ESPERA (falta material / cita)
- FINALIZADA
- CANCELADA

Flujo:
Backoffice crea OT -> asigna técnico -> técnico inicia -> consume materiales/instala equipos -> evidencia/firma -> solicita cierre -> supervisor cierra (o cierre automático si reglas cumplidas).

Reglas:
- No cerrar si falta lo obligatorio según tipo OT:
  - instalación: firma + evidencia + equipo instalado (si aplica)
  - mantenimiento: evidencia opcional, checklist mínimo
  - reparación: evidencia de problema y solución

3.7 Proyecto
Proyecto agrupa OTs y consumo.
Campos:
- nombre simple
- descripción
- estado (PLANIFICADO/EN_EJECUCION/PAUSADO/FINALIZADO/CANCELADO)
- fechas
- responsable
- almacén proyecto (opcional)
- etapas/tareas (opcional pero recomendado)

Etapas/Tareas:
- etapa: nombre, orden, %peso
- tarea/subtarea: nombre, estado, responsable

Avance proyecto:
- % por OTs finalizadas o por tareas completadas (define 1 método y úsalo).

3.8 Activo fijo y equipos serializados
Tipos:
- HERRAMIENTA (taladro, medidor)
- EQUIPO (modem, ONT, router)

Estados:
- DISPONIBLE
- ASIGNADO_A_TECNICO
- INSTALADO_EN_CLIENTE
- EN_REPARACION
- DANIADO
- PERDIDO
- BAJA

Flujos:
- Asignar activo a técnico
- Devolver activo a almacén
- Instalar equipo en cliente desde OT
- Retirar equipo de cliente
- Reemplazo:
  - registrar equipo saliente (retirado + dañado o a reparación)
  - registrar equipo entrante (instalado)
  - bloquear cierre si no existe el par

Reparación:
- enviar a reparación, recibir, resultado reparable/no reparable

3.9 Conteo físico vs sistema
Flujo:
- Iniciar conteo (ABIERTA) => snapshot sistema
- Captura físico
- Cerrar (CERRADA)
- Aprobar (APROBADA)
- Aplicar (APLICADA) => genera AJUSTE_CONTEO en kardex y actualiza stock

Reglas:
- Motivo obligatorio para diferencias.
- Diferencias grandes requieren comentario.

3.10 Reportes (operativos y gerenciales)
Definir reportes con filtros y columnas claras (sin adornos).
Operativos:
- Stock por almacén
- Stock bajo
- Kardex por producto/almacén
- Transferencias pendientes y confirmadas
- Consumo por OT
- Consumo por técnico
- Devoluciones y merma
- Conteos y diferencias
- Activos por estado (dañado, reparación, perdido)
Gerenciales:
- Consumo por proyecto y costo
- Avance proyectos
- OTs por estado y tiempos (SLA)
- Liquidación consignación por proveedor

========================================================
SECCIÓN 4 — UI/UX (SIMPLE + EFICIENTE + BONITO)
========================================================
Diseño:
- Blanco/gris/negro + rojo (acción) + verde (ok)
- Tablas tipo Excel: filtro, búsqueda, paginación server
- Acciones principales visibles y claras
- En móvil: flujo de 3 pasos (escanear->cantidad->guardar)

Pantallas web (mínimas) con lo que contienen:
- Login
- Inicio (KPIs + acciones rápidas)
- OT (lista, detalle)
- Proyectos (lista, detalle)
- Inventario (stock/kardex/entradas/salidas)
- Transferencias (crear/confirmar)
- Consignación (stock/liquidaciones)
- Activos (lista/detalle/historial)
- Conteo (sesiones/captura/resumen)
- Reportes (filtros + tabla + export)
- Administración (usuarios/roles/permisos/catálogos)

Pantallas móvil:
- Mis OT
- Detalle OT (checklist, consumo, activos, evidencia, firma, cerrar)
- Mi almacén (camioneta)
- Captura conteo (si rol aplica)

========================================================
SECCIÓN 5 — MODELO DE DATOS SQL SERVER (TABLAS + CAMPOS + KEYS)
========================================================
Debes entregar:
- Diagrama lógico (texto)
- Tablas con campos y tipos SQL Server
- PK/FK
- índices
- constraints
- auditoría

Tablas obligatorias (MANDATORIO: Prefijo Inv_):
SEGURIDAD
- Inv_seg_usuarios
- Inv_seg_roles
- Inv_seg_permisos (modulo, accion)
- Inv_seg_usuario_roles
- Inv_seg_rol_permisos
- Inv_seg_refresh_tokens
- Inv_seg_auditoria_eventos

CATÁLOGOS
- Inv_cat_proveedores
- Inv_cat_almacenes (idPadre, tipo, responsableUsuarioId, proyectoId opcional, tecnicoUsuarioId opcional)
- Inv_cat_categorias_producto
- Inv_cat_productos (codigo, nombre, unidad, esSerializado bit, costo decimal(18,2), activo bit)
- Inv_cat_producto_codigos (barcode/qr)
- Inv_cat_clientes
- Inv_cat_tipos_ot (reglas: requiereFirma bit, requiereEvidencia bit, requiereEquipoSerializado bit)
- Inv_cat_motivos_movimiento (MERMA, AJUSTE, DAÑO, DEVOLUCION, etc.)

PROYECTOS Y OT
- Inv_ope_proyectos
- Inv_ope_proyecto_tecnicos
- Inv_ope_proyecto_etapas
- Inv_ope_proyecto_tareas
- Inv_ope_ot
- Inv_ope_ot_checklist
- Inv_ope_ot_evidencias
- Inv_ope_ot_firmas
- Inv_ope_ot_eventos (timeline)
- Inv_ope_ot_consumo (detalle de consumos)
- Inv_ope_ot_activos (instalado/retirado/reemplazo)

INVENTARIO CONSUMIBLE
- Inv_inv_stock (almacenId, productoId, propietarioTipo, proveedorId null, cantidad)
- Inv_inv_movimientos (header)
- Inv_inv_movimiento_detalle (detail)

TRANSFERENCIAS
- Inv_inv_transferencias
- Inv_inv_transferencia_detalle

CONSIGNACIÓN
- Inv_inv_liquidacion_consignacion
- Inv_inv_liquidacion_consignacion_det

ACTIVOS SERIALIZADOS
- Inv_act_activos (serial unique, tipo, estado, almacenActualId, tecnicoActualId, clienteActualId null)
- Inv_act_movimientos (historial)
- Inv_act_reparaciones

CONTEO
- Inv_inv_conteos
- Inv_inv_conteo_detalle

Índices obligatorios:
- Inv_inv_stock(almacenId, productoId, propietarioTipo, proveedorId)
- Inv_inv_movimientos(fecha, tipoMovimiento, almacenOrigenId, almacenDestinoId)
- Inv_ope_ot(tecnicoId, estado, fechaAsignacion)
- Inv_act_activos(serial) UNIQUE
- Inv_inv_conteos(almacenId, estado, fechaInicio)

========================================================
SECCIÓN 6 — STORED PROCEDURES / CONSULTAS BASE (Prefijo Inv_sp_)
========================================================
Debes escribir SP base (SQL Server) con prefijo Inv_sp_ para operaciones críticas:

Inventario:
- Inv_sp_inv_stock_obtener (paginado + filtros)
- Inv_sp_inv_kardex_obtener (por producto/almacen/fechas)
- Inv_sp_inv_movimiento_registrar (entrada/salida/merma/devolución) => transaccional
- Inv_sp_inv_transferencia_crear
- Inv_sp_inv_transferencia_confirmar => transaccional (2 movimientos + actualizar stock)
- Inv_sp_inv_conteo_iniciar (snapshot)
- Inv_sp_inv_conteo_capturar
- Inv_sp_inv_conteo_aprobar
- Inv_sp_inv_conteo_aplicar => transaccional (ajustes + stock)

OT/Proyecto:
- Inv_sp_ot_crear
- Inv_sp_ot_asignar_tecnico
- Inv_sp_ot_iniciar
- Inv_sp_ot_consumir_material => llama sp de movimiento
- Inv_sp_ot_activo_reemplazar => transaccional (activos + ot_activos)
- Inv_sp_ot_cerrar (validaciones por tipo OT)
- Inv_sp_proyecto_resumen (avance + costos)

Activos:
- Inv_sp_activo_asignar_tecnico
- Inv_sp_activo_devolver
- Inv_sp_activo_enviar_reparacion
- Inv_sp_activo_recibir_reparacion
- Inv_sp_activo_dar_baja (requiere aprobación)

Reportes:
- Inv_sp_rep_stock_bajo
- Inv_sp_rep_consumo_por_ot
- Inv_sp_rep_consumo_por_tecnico
- Inv_sp_rep_consumo_por_proyecto
- Inv_sp_rep_consignacion_liquidacion
- Inv_sp_rep_activos_estado
- Inv_sp_rep_ot_sla_tiempos

Debes incluir:
- parámetros
- validaciones
- transacciones
- control de concurrencia (UPDLOCK/HOLDLOCK donde aplique)
- manejo de errores TRY/CATCH

========================================================
SECCIÓN 7 — REPORTES (DETALLE)
========================================================
Para cada reporte define:
- filtros
- columnas
- tabla fuente
- SP sugerido
- índices necesarios

========================================================
SECCIÓN 8 — PLAN 0–100 (CHECKLIST)
========================================================
Crea roadmap por fases con tareas numeradas:
Fase 1: Auth + roles + catálogos
Fase 2: Almacenes + inventario stock + kardex
Fase 3: OT + consumo + móvil
Fase 4: Transferencias
Fase 5: Conteo físico
Fase 6: Consignación
Fase 7: Activos + reemplazos + reparaciones
Fase 8: Reportes
Fase 9: Hardening (logs, backup, performance)
Fase 10: UAT + salida

Cada tarea:
- [ ] checkbox
- Prioridad Alta/Media/Baja
- Dependencias
- DoD

========================================================
SECCIÓN 9 — PENDIENTES POR CONFIRMAR
========================================================
Lista corta:
- ¿Firma siempre o depende de tipo OT?
- ¿Evidencia requerida siempre o por tipo OT?
- ¿Se requiere lote/vencimiento? (por defecto no)
- ¿Se requiere impresión? (por defecto no)
- ¿SLA por tipo OT? (tiempo máximo)
- ¿Activos consignados también?

========================================================
SECCIÓN 10 — ESTILO DE DESARROLLO (BACKEND)
========================================================
El desarrollo del backend seguirá los estándares de éxito de proyectos anteriores del usuario, priorizando rendimiento, seguridad y mantenibilidad.

10.1 Arquitectura y Tecnologías
- Framework: NestJS (REST API).
- Base de Datos: SQL Server.
- Acceso a Datos: Repository Pattern. Se prohíbe el uso de ORMs pesados como TypeORM para lógica compleja. Se utilizará el driver nativo 'mssql' a través de un BaseRepo optimizado.
- SP-First: Toda la lógica de negocio crítica, validaciones de stock y consultas complejas DEBEN residir en Stored Procedures.
- Transaccionalidad: Uso obligatorio de transacciones (conTransaccion) para asegurar la integridad referencial y de inventario.
- Auditoría: Registro automático de consultas lentas (>1000ms) en Inv_p_SlowQueries.

10.2 Reglas de Nomenclatura (CRÍTICO)
- Prefijo Único: TODA tabla, vista, función o procedimiento almacenado de este proyecto DEBE iniciar con el prefijo "Inv_".
- Ejemplo: Inv_seg_usuarios, Inv_cat_productos, Inv_sp_ot_crear.
- Propósito: Aislamiento total en la base de datos "Bdplaner" para no interferir con otros proyectos existentes.
- Formato: snake_case para nombres de tablas y campos.

10.3 Manejo de Errores y Seguridad
- TRY/CATCH en todos los SPs con retorno de códigos de error estandarizados.
- Validación de permisos (RBAC) en la capa de Service antes de llamar al Repository.
- Sanitización de parámetros integrada en el BaseRepo.

========================================================
ENTREGABLE FINAL
========================================================
Genera el documento completo en español, con:
- estructura clara
- tablas SQL Server (campos y tipos) con prefijo Inv_
- SP base (al menos los críticos) con prefijo Inv_
- UI detallada por página
- flujos completos
- reportes
- checklist 0–100

No omitas consignación, conteo físico, activos serializados y reemplazo en OT.
No uses azul en diseño.